\documentclass[paper=a4, fontsize=11pt]{report} % A4 paper, 11pt font-size
\usepackage[margin=1in]{geometry} % 1 inch margin
\linespread{1} % single spacing

% --------------------
% Packages
% --------------------

\usepackage[T1]{fontenc} % Use 8-bit encoding that has 256 glyphs
\usepackage{fourier} % Use the Adobe Utopia font for the document - comment this line to return to the LaTeX default
\usepackage[english]{babel} % English language/hyphenation
\usepackage{amsmath,amsfonts,amsthm,amssymb} % Math packages
\usepackage{graphicx}
\usepackage{float} % Use [H] to force figure not float
\usepackage[section]{placeins} % prevent figures to appear in another section
\usepackage[vlined,lined,boxed,commentsnumbered]{algorithm2e}
\usepackage[colorlinks=true]{hyperref} % hyperlink

% \usepackage{sectsty} % Allows customizing section commands
% \allsectionsfont{\centering \normalfont\scshape} % Make all sections centered, the default font and small caps
%\setlength\parindent{0pt} % Removes all indentation from paragraphs - comment this line for an assignment with lots of text
% \usepackage{indentfirst} % indent the first line of a section

% Font Setting
% \usepackage{fontspec}
% \setmainfont{Times New Roman}
% \setmainfont{STIXGeneral}
% \setsansfont{Arial}
% \setmonofont{Courier New}

% \numberwithin{equation}{section} % Number equations within sections (i.e. 1.1, 1.2, 2.1, 2.2 instead of 1, 2, 3, 4)
% \numberwithin{figure}{section} % Number figures within sections (i.e. 1.1, 1.2, 2.1, 2.2 instead of 1, 2, 3, 4)
% \numberwithin{table}{section} % Number tables within sections (i.e. 1.1, 1.2, 2.1, 2.2 instead of 1, 2, 3, 4)

% --------------------
% Title
% --------------------

\title{Dell SonicWall Visualization Report}
\author{
Wanzhang Sheng\\
Kaiming Yang\\
Michelle Gao\\
Xi Han
}

\begin{document}
\maketitle


\chapter{Introduction} % (fold)
\label{cha:introduction}
% The introduction section should contain two subsections: a project overview, and a team overview. The project overview should provide a high-level description of your project, including a summary of your deliverables.
% The team overview should include a biography of each team member and your sponsor (or sponsor's company). Include details relevant to your project, including the skills and experience of each team member.
% Aim for 3 to 5 paragraphs for your project overview. There should be at least one paragraph per person in the team overview.



% chapter introduction (end)


\chapter{Specification} % (fold)
\label{cha:specification}
% The specification section should contain two subsections: functionality requirements and design. These sections should come primarily from your draft specification.
% Please see the Draft Specification assignment for additional details on what to include in these sections. Aim for 2 or more pages for this section.



% chapter specification (end)


\chapter{Implementation} % (fold)
\label{cha:implementation}
% The implementation section should contain three subsections: results, obstacles, and a timeline. The results section is where you discuss in detail your final results. The specification discusses what you wanted to complete, whereas this section discusses what you actually completed. Be specific about the functionality you implemented and the environments/tools you used.
% The obstacles section is your chance to discuss the obstacles your team encountered that slowed your progress. Discuss what happened, how you addressed the issue, and how this affected your timeline.
% The timeline section should show the actual dates you completed major milestones, and projected dates to complete the unfinished portions of your specification and test plan.
% Aim for 2 or more pages for this section.



% chapter implementation (end)


\chapter{Test Plan} % (fold)
\label{cha:test_plan}
% The test plan section should contain two subsections: objectives and scenarios. These sections should come primarily from your draft test plan.
% Please see the Draft Test Plan assignment for additional details on what to include in these sections. Aim for 2 or more pages for this section.



% chapter test_plan (end)


\chapter{Deliverables} % (fold)
\label{cha:deliverables}
% The deliverables section should contain three subsections: primary deliverables, other deliverables, and documentation.
% For your primary deliverables, list the specific files containing the code you created for the primary functionality of this project. Include specific filenames and directory structure, and statistics such as the number of files, file size of these files combined, and SLOC (source-lines-of-code) for the code you provided. Use any tool you want for the SLOC (I suggest sloccount (Links to an external site.) or the metrics (Links to an external site.) Eclipse page). Do not include data files in this section.
% The other deliverables section should contain any data files (XML, JSON, CSV, etc.) being provided, test code, or deliverables that do not fall under another category. Include statistics for these deliverables if possible. For example, "We include ## MB of data files, and ## SLOC of test code with our project."
% The documentation section should describe where to find documentation for your project. List the specific files that provide user documentation (for using your project), and where to find developer documentation (for extending/maintaining your project).
% If you do not provide documentation, state a strong case for why you do not provide documentation. If you do not have a strong case, you will be docked points on your final report.
% Aim for 2 or more pages for this section.



% chapter deliverables (end)


\end{document}
