\documentclass[paper=a4, fontsize=11pt]{scrartcl} % A4 paper and 11pt font size
\usepackage{assignment}
\settitle{Status Review 1}
\setauthor{Wanzhang Sheng}
\setcourse{Master Project}

\begin{document}

\maketitle % Print the title

Our team name is GreatFireWall (GFW).

% What contact have you had with your sponsor? Do you have regular meetings setup? Do you feel you understand the project requirements?
\section{Sponsor Update} % (fold)
\label{sec:sponsor_update}
Our Dell sponsor, Carrie Gates, is currently uncontactable.
We have meeting every Wednesday with Sophie.
We got the basic ideas of the project requirements and will address more details as the project going.
% section sponsor_update (end)


% Discuss how you plan to work as a team throughout the semester. How will you use the classtime? Will you meet weekly outside of classtime? How do you plan to divvy responsibilities and resolve conflicts? And, give your team a name!
\section{Team Update} % (fold)
\label{sec:team_update}
We plan to have a short meeting every weekdays, talk about the progress and problem (if any). Class time will be used for the daily meeting on Tuesday and Thursday, as well every Wednesday at 1:00pm before we meet our sponsor.

A vote will be hold to make the decision if it's hard to reach agreement,. If vote still fails, the person who takes action will make the decision.

Since we can't get the real data now. Sophie assigned us some tasks to get familiar with d3 library for the last and next weeks.
% section team_update (end)


% Discuss the project itself. Explain the project, what languages/tools you plan to use, what resources you need to succeed, what you feel the milestones and deliverables will be, and any obstacles you anticipate encountering. If your project does not have a name, give it a name!
\section{Project Discussion} % (fold)
\label{sec:project_discussion}
The project is to build stand-alone prototype visualizations of Dell SonicWall firewall data that could be integrated into Dell SonicWall Analyzer and Scrutinizer.

The languages we plan to use is JavaScript with d3.js (or p5.js) libraries for visualization, and any other JavaScript libraries to parse and filter the data.

The milestones are acquiring techniques of data visualization, learning SonicWall data format. The deliverables will be integrating the prototype into visualizations which are hopefully informative and attractive.

The obstacles we anticipate to encounter is to present informative visualization and the deliverability for Supercomputing in November.
% section project_discussion (end)

\end{document}
